%!TEX program = xelatex
\documentclass{article}
\usepackage{amsmath}
\usepackage{ctex}
\title{Quantum Field Theory: Homework}
\author{BenjaminDbb}
\begin{document}
\maketitle

\section{}
\bf{(a)}
\begin{align}
F^{\mu \nu} 			&= \partial^{\mu} A^{\nu} - \partial^{\nu} A^{\mu}  \nonumber \\
F_{\mu \nu} F^{\mu \nu} &= ({\partial_{\mu} A_{\nu} - \partial_{\nu} A_{\mu}}) ({\partial^{\mu} A^{\nu} - \partial^{\nu} A^{\mu}}) \nonumber \\
						&= \partial_{\mu} A_{\nu} \partial^{\mu} A^{\nu} - \partial_{\nu} A_{\mu} \partial^{\mu} A^{\nu} - 
						\partial_{\mu} A_{\nu} \partial^{\nu} A^{\mu} + \partial_{\nu} A_{\mu} \partial^{\nu} A^{\mu} \nonumber
\end{align}
hello year in my heart!
且在上述表达式中$\mu \: \nu$都

\begin{equation}
F_{\mu \nu} F^{\mu \nu} = 2(\partial_{\mu} A_{\nu} \partial^{\mu} A^{\nu} - \partial_{\nu} A_{\mu} \partial^{\mu} A^{\nu})
\end{equation}

由经典无源电磁场的作用量表达式 $$S=\int d^4 x \mathcal{L} =\int d^4 x ( -\frac{1}{4} F_{\mu \nu} F^{\mu \nu} )$$ 及由欧拉-拉格朗日公式$$\frac{\partial \mathcal{L}}{\partial A_{\beta}} - \partial_{\lambda} \frac{\partial \mathcal{L}}{\partial (\partial_{\lambda} A_{\beta})} = 0 $$
且在拉格朗日量表达式中$A^{\mu}$与$\partial^{\mu} A^{\nu}$是独立变量。且拉格朗日量$$\mathcal{L} =  -\frac{1}{4} F_{\mu \nu} F^{\mu \nu} $$ 仅仅与 $\partial^{\mu} A^{\nu}$ 有关 \\
则EOM为
$$\partial_{\lambda} \left[ \frac{\partial(\partial_{\mu} A_{\nu} \partial^{\mu} A^{\nu} - \partial_{\nu} A_{\mu} \partial^{\mu} A^{\nu})}{\partial(\partial_{\lambda} A_{\beta})} \right] = 0 \nonumber $$
展开得
$$
\partial_{\lambda} \left[ \frac{\partial (\partial_{\mu} A_{\nu})}{\partial(\partial_{\lambda} A_{\beta})} \partial^{\mu} A^{\nu} + \partial_{\mu} A_{\nu} \frac{\partial (\partial^{\mu} A^{\nu})}{\partial (\partial_{\lambda} A_{\beta})} - \frac{\partial (\partial_{\nu} A_{\mu})}{\partial (\partial_{\lambda} A_{\beta})} \partial^{\mu} A^{\nu} - \partial_{\mu} A_{\nu} \frac{\partial (\partial^{\mu} A^{\nu})}{\partial (\partial_{\lambda} A_{\beta})} \right] = 0  \nonumber 			   
$$
其中以第一项为例进行计算
\begin{align}
\frac{\partial (\partial_{\mu} A_{\nu})}{\partial(\partial_{\lambda} A_{\beta})} \partial^{\mu} A^{\nu} &=
\frac{\partial (\frac{\partial}{\partial x^{\mu}} A_{\nu})}{\partial(\frac{\partial}{\partial x^{\lambda}} A_{\beta})} \partial^{\mu} A^{\nu} \nonumber \\
	&= \delta_{\nu}^{\phantom{\nu} {\beta}} \delta^{\lambda}_{\phantom{\lambda} {\mu}} \partial^{\mu} A^{\nu}  \nonumber \\
	&= \partial^{\lambda} A^{\beta} \nonumber
\end{align}
同理地,将其他三项进行同样的计算可得EOM为
\begin{align}
2\partial_{\lambda} \partial^{\lambda} A^{\beta} - 2\partial_{\lambda} \partial^{\beta} A^{\lambda} &=0 \nonumber \\
2\partial_{\lambda} (\partial^{\lambda} A^{\beta} - \partial^{\beta} A^{\lambda} ) &=0 \nonumber \\
2\partial_{\lambda} F^{\lambda \beta} &=0 \nonumber \\
\partial_{\lambda} F^{\lambda \beta} &=0 \nonumber 
\end{align}
是Maxwell's equations的其中一个方程\\
根据题意,$E^i = -F^{0i}$且$\epsilon^{ijk}B^{k} = -F^{ij}$得
$$
F^{\lambda \beta} = 
\left[ 
	\begin{matrix}
	0 & -E^1 & -E^2 & -E^3 \\
	E^1 & 0	& -B^3 & B^2 \\
	E^2 & B^3 & 0 & -B^1 \\
	E^3 & -B^2 & B^1 & 0 \\	
	\end{matrix}
\right]
$$
$$
\partial_{\lambda} = \left[ \partial^{0}, -\partial^{1},-\partial^{2},-\partial^{3}\right]
$$
带入场方程的表达式中,可得
\begin{align}
&\partial_{\lambda} F^{\lambda \beta} \nonumber \\&= (-\partial^{1}E^{1}-\partial^{2}E^{2}-\partial^{3}E^{3},
	-\partial^{0}E^{1}-\partial^{2}B^{3}+\partial^{3}B^{2},
	-\partial^{0}E^{2}-\partial^{3}B^{1}+\partial^{1}B^{3},
	-\partial^{0}E^{3}-\partial^{1}B^{2}+\partial^{2}B^{1}) \nonumber \\ &=
	(0,0,0,0) \nonumber
\end{align}
可得:
\begin{equation*}
\left\{ \begin{aligned}
\nabla \cdot \mathbf{E} =0,\\
\nabla \times \mathbf{B} = \frac{\partial}{\partial t} \mathbf{E} 
\end{aligned}
\right.
\end{equation*}

又因为$F^{\mu \nu} = \partial^{\mu} A^{\nu} - \partial^{\nu} A^{\mu} $所以,自动满足一个轮换式:
\begin{align}
&\partial^{\lambda} F^{\mu \nu} + \partial^{\mu} F^{\nu \lambda} +\partial^{\nu} F^{\lambda \mu} \nonumber \\
&= \partial^{\lambda}(\partial^{\mu} A^{\nu} - \partial^{\nu} A^{\mu}) + \partial^{\mu}  (\partial^{\nu} A^{\lambda} - \partial^{\lambda} A^{\nu}) +\partial^{\nu}  (\partial^{\lambda} A^{\mu} - \partial^{\mu} A^{\lambda}) \nonumber \\
&=0 \nonumber 
\end{align}
通过构造一个新的张量:$\widetilde{F}^{\mu \nu} = \frac{1}{2} \epsilon^{\mu \nu \rho \sigma} F_{\rho \sigma} $\\
将上面得轮换表达式可以变为$\partial_{\mu} \widetilde{F}^{\mu \nu} =0$
且
$$
\widetilde{F}^{\mu \nu} = 
\left[ 
	\begin{matrix}
	0 & -B^1 & -B^2 & -B^3 \\
	B^1 & 0	& E^3 & -E^2 \\
	B^2 & -E^3 & 0 & E^1 \\
	B^3 & E^2 & -E^1 & 0 \\	
	\end{matrix}
\right]
$$
类似与上面一样的计算,可以得到:
\begin{equation*}
\left\{ \begin{aligned}
\nabla \cdot \mathbf{B} =0,\\
\nabla \times \mathbf{E} =- \frac{\partial}{\partial t} \mathbf{B} 
\end{aligned}
\right.
\end{equation*}
\bf{(b)}

Nother定理在导出守恒流公式的时候,也定义能动张量有如下表达式
$$
T^{\mu}_{\phantom{\mu}\nu} = \frac{\partial \mathcal{L}}{\partial (\partial_{\mu}\phi)}  \partial_{\nu} \phi- \delta^{\mu}_{\phantom{\mu}\nu} \mathcal{L}
$$
将电磁场的情况进行带入,可以得到
$$
T^{\mu}_{\phantom{\mu}\nu} = \frac{\partial \mathcal{L}}{\partial (\partial_{\mu}A_{\beta})}  \partial_{\nu} A_{\beta}- \delta^{\mu}_{\phantom{\mu}\nu} \mathcal{L}
$$
进一步改写成
$$
T^{\mu \nu} = \frac{\partial \mathcal{L}}{\partial (\partial_{\mu}A_{\beta})}  \partial^{\nu} A_{\beta}- \delta^{\mu \nu} \mathcal{L}
$$
根据在(a)中的计算可得$\frac{\partial \mathcal{L}}{\partial (\partial_{\mu}A_{\beta})}  = F^{\beta \mu}$,所以$$T^{\mu \nu} = F^{\beta \mu} \partial^{\nu} A_{\beta}- \delta^{\mu \nu} \mathcal{L}$$
是该电磁场理论下下的能动张量

因为对于能动张量,必有如下关系成立$\partial_{\mu} T^{\mu \nu} =0 $,如果我们要去添加一项,使关系不变。\\
使用关系式反对称关系$K^{\lambda \mu \nu} = - K^{\mu \lambda \nu}$ 可得:
\begin{align}
\partial_{\mu} \hat{T}^{\mu \nu} &=\partial_{\mu} (T^{\mu \nu} + \partial_{\lambda} K^{\lambda \mu \nu}) \nonumber \\
				&=\partial_{\mu} T^{\mu \nu} + \partial_{\mu} \partial_{\lambda} K^{\lambda \mu \nu} \nonumber \\
				&= 0 + \partial_{\mu} \partial_{\lambda} K^{\lambda \mu \nu} \nonumber \\
				&= \frac{1}{2} (\partial_{\mu} \partial_{\lambda} K^{\lambda \mu \nu} + \partial_{\lambda} \partial_{\mu} K^{\mu \lambda \nu} ) \nonumber \\
				&= \frac{1}{2} (\partial_{\mu} \partial_{\lambda} K^{\lambda \mu \nu} + \partial_{\mu} \partial_{\lambda} K^{\mu \lambda \nu} ) \nonumber \\
				&=\frac{1}{2} \partial_{\mu} \partial_{\lambda} (K^{\lambda \mu \nu} + K^{\mu \lambda \nu}) \nonumber \\
				&=0 \nonumber
\end{align}
如果我们构造$K^{\lambda \mu \nu} = F^{\mu \lambda} A^{\nu}$可以使能动张量变为对称张量,来满足题设条件。\\
我们有:
\begin{align}
\hat{T}^{\mu \nu} &= T^{\mu \nu} + \partial_{\beta} K^{\beta \mu \nu} \nonumber \\
				&= F^{\beta \mu} \partial^{\nu} A_{\beta}- \delta^{\mu \nu} \mathcal{L} + \partial_{\beta}( F^{\mu \beta} A^{\nu})\nonumber \\
				&= F^{\beta \mu} \partial^{\nu} A_{\beta}- \delta^{\mu \nu} \mathcal{L} + \partial_{\beta} F^{\mu \beta} A^{\nu} + F^{\mu \beta} \partial_{\beta}A^{\nu} \nonumber 
\end{align}
其中 拉格朗日量项明显一定是对称的,后面$\partial_{\beta} F^{\mu \beta} A^{\nu} = - \partial_{\beta} F^{\beta \mu} A^{\nu} = 0$,所以,最后讨论的表达式为:
$$
F^{\beta \mu} \partial^{\nu} A_{\beta} + F^{\mu \beta} \partial_{\beta}A^{\nu} = F^{\beta \mu} \partial^{\nu} A_{\beta} - F^{\beta \mu} \partial_{\beta}A^{\nu} =  F^{\beta \mu}(\partial^{\nu} A_{\beta} -\partial_{\beta}A^{\nu} ) 
= F^{\beta \mu} F^{\nu}_{\phantom{\nu} \beta}
$$
则若$F^{\beta \mu} F^{\nu}_{\phantom{\nu} \beta}$为对称的,即题设成立。
在这种情况下$$T^{\mu \nu} = F^{\beta \mu} F^{\nu}_{\phantom{\nu} \beta} - \delta^{\mu \nu} \mathcal{L}$$
写出$$
F_{\lambda \beta} = 
\left[ 
	\begin{matrix}
	0 & E^1 & E^2 & E^3 \\
	-E^1 & 0	& -B^3 & B^2 \\
	-E^2 & B^3 & 0 & -B^1 \\
	-E^3 & -B^2 & B^1 & 0 \\	
	\end{matrix}
\right]
\quad
F^{\lambda}_{\phantom{\lambda} \beta} = 
\left[ 
	\begin{matrix}
	0 & E^1 & E^2 & E^3 \\
	E^1 & 0	& B^3 & -B^2 \\
	E^2 & -B^3 & 0 & B^1 \\
	E^3 & B^2 & -B^1 & 0 \\	
	\end{matrix}
\right]
$$
带入后可证明$F^{\beta \mu} F^{\nu}_{\phantom{\nu} \beta}$是对称的。\\
且其00分量$$T^{00} =  F^{\beta 0} F^{0}_{\phantom{0} \beta} - \delta^{0 0} \mathcal{L}$$
且
\begin{align}
\mathcal{L} &= - \frac{1}{4} F^{\mu \nu} F_{\mu \nu} \nonumber \\
			&= - \frac{1}{4} (F^{ 0 0} F_{0 0 } + F^{0 i} F_{0 i}+ F^{i 0} F_{i 0}+ F^{i j} F_{i j}) \nonumber \\
			&= - \frac{1}{4} (0- 2\mathbf{E}^2 +(- \epsilon^{ijk} B^{k} )(- \epsilon_{ijm} B_{m})) \nonumber \\
			&= - \frac{1}{4} (-2\mathbf{E}^2 + 2B^{k}B_{m} \delta^{k}_{m}) \nonumber \\
			&=  \frac{1}{2} (\mathbf{E}^2 -\mathbf{B}^2) \nonumber
\end{align}
\begin{align}
 F^{\beta 0} F^{0}_{\phantom{0} \beta}
		&= F^{0 0} F^{0}_{\phantom{0} 0} +F^{i 0} F^{0}_{\phantom{0} i} \nonumber \\
		&= \mathbf{E}^2 \nonumber
\end{align}
$$T^{00} =  F^{\beta 0} F^{0}_{\phantom{0} \beta} - \delta^{0 0} \mathcal{L}= \frac{1}{2} (\mathbf{E}^2 +\mathbf{B}^2)$$
\begin{align}
S^{k} = T^{0 k} &=  F^{\beta 0} F^{k}_{\phantom{k} \beta} \nonumber \\
	&=  F^{0 0} F^{0}_{\phantom{0} 0} +  F^{j 0} F^{0}_{\phantom{0} j} + F^{0 0} F^{i}_{\phantom{i} 0} +  F^{j 0} F^{i}_{\phantom{i} j} \nonumber \\
	&= E^{j} \epsilon_{ijk}B^{k} \nonumber \\
\mathbf{S} &= \mathbf{E} \times \mathbf{B} \nonumber
\end{align}
\section{}
\bf{(a)}
$$S=\int d^4 x (\partial_{\mu} \phi^{*} \partial^{\mu} \phi - m^2 \phi^{*}\phi)$$
其中,
\begin{align}
\mathcal{L} &= \partial_{\mu} \phi^{*} \partial^{\mu} \phi - m^2 \phi^{*}\phi \nonumber \\
			&=\dot{\phi^{*}} \dot{\phi} - \nabla \phi^{*} \cdot \nabla \phi - m^2 \phi^{*}\phi \nonumber
\end{align}
与之对应的广义动量有
$$
\pi = \frac{\partial \mathcal{L}}{\partial \dot{\phi}} = \dot{\phi^{*}}
\qquad 
\pi^{*} = \frac{\partial \mathcal{L}}{\partial \dot{\phi^{*}}} = \dot{\phi}
$$
\begin{align}
\mathcal{H} &= \pi \dot{\phi} + \pi^{*} \dot{\phi^{*}} - \mathcal{L} \nonumber \\
			&= \dot{\phi^{*}} \dot{\phi} + \dot{\phi} \dot{\phi^{*}} - \mathcal{L} \nonumber \\
			&= \dot{\phi^{*}} \dot{\phi} + \nabla \phi^{*} \cdot \nabla \phi + m^2 \phi^{*}\phi \nonumber \\
		H   &= \int d^3 x \mathcal{H} \nonumber \\
			&= \int d^3 x \left( \dot{\phi^{*}} \dot{\phi} + \nabla \phi^{*} \cdot \nabla \phi + m^2 \phi^{*}\phi \right) \nonumber \\
			&= \int d^3 x \left( \pi^{*} \pi+ \nabla \phi^{*} \cdot \nabla \phi + m^2 \phi^{*}\phi \right) \nonumber 
\end{align}
通过将复标量场进行如下构造:
$$\phi(x) = \frac{1}{\sqrt{2}} (\phi_1(x) + i \phi_2(x))$$
$$\phi^{*}(x) = \frac{1}{\sqrt{2}} (\phi_1(x) - i \phi_2(x))$$
我们对$\phi_1(x)$与$\phi_2(x)$已经有关于实标量场的对易关系,满足 \\
$$
[\phi_{1}(\mathbf{x},t), \phi_{1}(\mathbf{x^{\prime}},t)] = [\phi_{2}(\mathbf{x},t), \phi_{2}(\mathbf{x^{\prime}},t)]=0 
$$
$$
[\phi_{1}(\mathbf{x},t), \pi_{1}(\mathbf{x^{\prime}},t)] = [\phi_{2}(\mathbf{x},t), \pi_{2}(\mathbf{x^{\prime}},t)] = i\delta(\mathbf{x} - \mathbf{x^{\prime}})
$$
带入可以得到复标量场的正则对易关系为:
$$
 [\phi(\mathbf{x},t), \phi(\mathbf{x^{\prime}},t)] = [\pi(\mathbf{x},t), \pi(\mathbf{x^{\prime}},t)] 
=[\phi^{*}(\mathbf{x},t), \phi^{*}(\mathbf{x^{\prime}},t)] = [\pi^{*}(\mathbf{x},t), \pi^{*}(\mathbf{x^{\prime}},t)] =0 
$$
$$
[\phi(\mathbf{x},t), \pi(\mathbf{x^{\prime}},t)] = [\phi^{*}(\mathbf{x},t), \pi^{*}(\mathbf{x^{\prime}},t)] = i\delta(\mathbf{x} - \mathbf{x^{\prime}})
$$
Heisenberg EOM : $$i \frac{\partial}{\partial t} \hat{\mathcal{O}}= \left[ \hat{\mathcal{O}},\hat{H} \right]$$

\begin{align} \:
i \frac{\partial}{\partial t} \hat{\pi}(\mathbf{x},t) 
	&=\left[ \hat{\pi}(\mathbf{x},t),\hat{H} \right] \nonumber \\
	&=\left[ \hat{\pi}(\mathbf{x},t),\int d^3 x^{\prime} \left(\hat{\pi}^{*}(\mathbf{x^{\prime}},t) \hat{\pi}(\mathbf{x^{\prime}},t)+ \nabla^{\prime} \hat{\phi}^{*}(\mathbf{x^{\prime}},t) \cdot \nabla^{\prime} \hat{\phi}(\mathbf{x^{\prime}},t) + m^2 \hat{\phi}^{*}(\mathbf{x^{\prime}},t)\hat{\phi}(\mathbf{x^{\prime}},t) \right) \right] \nonumber \\
	&=\int d^3 x^{\prime} \left[ \hat{\pi}(\mathbf{x},t),\nabla^{\prime} \hat{\phi}^{*}(\mathbf{x^{\prime}},t) \cdot \nabla^{\prime} \hat{\phi}(\mathbf{x^{\prime}},t) + m^2 \hat{\phi}^{*}(\mathbf{x^{\prime}},t)\hat{\phi}(\mathbf{x^{\prime}},t) \right] \nonumber \\
	&=\int d^3 x^{\prime} \: \nabla^{\prime} \hat{\phi}^{*}(\mathbf{x^{\prime}},t)  \left[ \hat{\pi}(\mathbf{x},t),\nabla^{\prime} \hat{\phi}(\mathbf{x^{\prime}},t) \right] \nonumber \\
	&+ \int d^3 x^{\prime}  \: m^2 \hat{\phi}^{*}(\mathbf{x^{\prime}},t) \left[ \hat{\pi}(\mathbf{x},t), \hat{\phi}(\mathbf{x^{\prime}},t) \right] \nonumber \\
	&= \int d^3 x^{\prime} \: \left(
			-i \nabla^{\prime} \delta(\mathbf{x}-\mathbf{x}^{\prime})
		\right)\nabla^{\prime} \hat{\phi}^{*}(\mathbf{x^{\prime}},t) \nonumber \\
	&+ \int d^3 x^{\prime} \: m^2 \hat{\phi}^{*}(\mathbf{x^{\prime}},t) (-i \delta(\mathbf{x}-\mathbf{x}^{\prime})) \nonumber \\
	&= -i\int d^3 x^{\prime} \: \left(
			\nabla^{\prime}(\nabla^{\prime} \hat{\phi}^{*}(\mathbf{x^{\prime}},t) \delta(\mathbf{x}-\mathbf{x}^{\prime})) - \nabla^{\prime 2} \hat{\phi}^{*}(\mathbf{x^{\prime}},t) \delta(\mathbf{x}-\mathbf{x}^{\prime})
	\right) \nonumber \\
	&=\nabla^{\prime} \hat{\phi}^{*}(\mathbf{x^{\prime}},t) \delta(\mathbf{x}-\mathbf{x}^{\prime}) |^{+\infty}_{-\infty} +i\int d^3 x^{\prime} \: \nabla^{\prime 2} \hat{\phi}^{*}(\mathbf{x^{\prime}},t) \delta(\mathbf{x}-\mathbf{x}^{\prime}) \nonumber \\
	&+ \int d^3 x^{\prime} \: m^2 \hat{\phi}^{*}(\mathbf{x^{\prime}},t) (-i \delta(\mathbf{x}-\mathbf{x}^{\prime})) \nonumber \\
	&=i\int d^3 x^{\prime} \: \nabla^{\prime 2} \hat{\phi}^{*}(\mathbf{x^{\prime}},t) \delta(\mathbf{x}-\mathbf{x}^{\prime}) \nonumber +\int d^3 x^{\prime} \: m^2 \hat{\phi}^{*}(\mathbf{x^{\prime}},t) (-i \delta(\mathbf{x}-\mathbf{x}^{\prime})) \nonumber \\
	&= i \left(
		\nabla^2 - m^2
	\right) \hat{\phi}^{*}(\mathbf{x},t) \nonumber \\
	&=i \frac{\partial}{\partial t} \hat{\pi}(\mathbf{x},t)
	= i \frac{\partial^2}{\partial t^2} \hat{\phi}^{*}(\mathbf{x},t) \nonumber 
\end{align}
可得K-G equation:
\begin{align}
(\frac{\partial^2}{\partial t^2} -\nabla^2 + m^2)\hat{\phi}^{*}(\mathbf{x},t) 	&=0 \nonumber \\
(\partial^{\mu} \partial_{\mu} +m^2) \hat{\phi}^{*}(\mathbf{x},t) 				&=0 \nonumber \\
(\partial^2 + m^2) \hat{\phi}^{*}(\mathbf{x},t) 								&=0 \nonumber
\end{align}
使用同样的方法计算
$$
i \frac{\partial}{\partial t} \hat{\pi}^{*}(\mathbf{x},t) = \left[ \hat{\pi}^{*}(\mathbf{x},t),\hat{H} \right] 
$$
也可得到$(\partial^2 + m^2) \hat{\phi}(\mathbf{x},t) =0$

\bf{(b)}
$$\phi(\mathbf{x},t) = \int \frac{d^3 p}{(2\pi)^3} \frac{1}{\sqrt{2E_\mathbf{p}}} \left(
a_{\mathbf{p}} e^{-ipx} +  b^{\dagger}_\mathbf{p} e^{ipx} \right) 
= \int d \widetilde{p}  \left(
a_{\mathbf{p}} e^{-i E_{\mathbf{p}}t} +  b^{\dagger}_{-\mathbf{p}} e^{i E_{\mathbf{p}}t} \right) 
e^{i \mathbf{p} \mathbf{x}}$$
$$\phi^{*}(\mathbf{x},t) = \int \frac{d^3 p}{(2\pi)^3}  \frac{1}{\sqrt{2E_\mathbf{p}}} \left(
a^{\dagger}_{\mathbf{p}} e^{ipx} +  b_\mathbf{p} e^{-ipx} \right)
=\int d \widetilde{p}  \left(
a^{\dagger}_{-\mathbf{p}} e^{i E_{\mathbf{p}}t} +  b_{\mathbf{p}} e^{-i E_{\mathbf{p}}t} \right) 
e^{i \mathbf{p} \mathbf{x}}$$
$$\pi^{*}(\mathbf{x},t)=\dot{\phi}(\mathbf{x},t) 
=i\int \frac{d^3 p}{(2\pi)^3} \sqrt{\frac{E_\mathbf{p}}{2}}\left(
- a_{\mathbf{p}} e^{-ipx} +  b^{\dagger}_\mathbf{p} e^{ipx} \right)
=\int d \widetilde{p}  (i E_{\mathbf{p}})\left(
- a_{\mathbf{p}} e^{-i E_{\mathbf{p}}t} +  b^{\dagger}_{-\mathbf{p}} e^{i E_{\mathbf{p}}t} \right) 
e^{i \mathbf{p} \mathbf{x}}$$
$$\pi(\mathbf{x},t)=\dot{\phi^{*}}(\mathbf{x},t) 
= i \int \frac{d^3 p}{(2\pi)^3}  \sqrt{\frac{E_\mathbf{p}}{2}} \left(
a^{\dagger}_{\mathbf{p}} e^{ipx} -  b_\mathbf{p} e^{-ipx} \right)
=\int d \widetilde{p} (i E_{\mathbf{p}}) \left(
a^{\dagger}_{-\mathbf{p}} e^{i E_{\mathbf{p}}t} -  b_{\mathbf{p}} e^{-i E_{\mathbf{p}}t} \right) 
e^{i \mathbf{p} \mathbf{x}}$$
带入表达式$H= \int d^3 x \left( \pi^{*} \pi+ \nabla \phi^{*} \cdot \nabla \phi + m^2 \phi^{*}\phi \right)$之中
得,且有表达式$\int d^3 x \: e^{i \mathbf{k} \mathbf{x}} = (2 \pi)^3 \delta(\mathbf{k})$和表达式$E_{\mathbf{p}}=E_{-\mathbf{p}}$得:
\begin{align}
H 	&= \int d^3 x \: d \widetilde{p} \: d \widetilde{k} \: e^{i(\mathbf{p} +  \mathbf{k}) \mathbf{x}} 		\left[ 	(i E_{\mathbf{p}})\left( 
		- a_{\mathbf{p}} e^{-i E_{\mathbf{p}}t} +  b^{\dagger}_{-\mathbf{p}} e^{i E_{\mathbf{p}}t} \right)
			(i E_{\mathbf{k}}) \left( 
a^{\dagger}_{-\mathbf{k}} e^{i E_{\mathbf{k}}t} -  b_{\mathbf{k}} e^{-i E_{\mathbf{k}}t} \right) \right] \nonumber \\
	&+\int d^3 x \: d \widetilde{p} \: d \widetilde{k} \: e^{i(\mathbf{p} +  \mathbf{k}) \mathbf{x}} \left[ 
	 ((i  \mathbf{p})(i  \mathbf{k}) +m^2 )\left(
		a_{\mathbf{p}} e^{-i E_{\mathbf{p}}t} +  b^{\dagger}_{-\mathbf{p}} e^{i E_{\mathbf{p}}t} \right) 
		\left(
		a^{\dagger}_{-\mathbf{k}} e^{i E_{\mathbf{k}}t} +  b_{\mathbf{k}} e^{-i E_{\mathbf{k}}t} \right) 
	\right] \nonumber \\
	&= \int d \widetilde{p} \: d \widetilde{k} \: (2 \pi)^3 \delta(\mathbf{k} + \mathbf{p})\left[ 
	 (E_{\mathbf{p}} E_{\mathbf{k}} -\mathbf{p} \mathbf{k} + m^2)
	 (a_{\mathbf{p}} a^{\dagger}_{-\mathbf{k}} e^{i(E_{\mathbf{k}}-E_{\mathbf{p}})t}+ b^{\dagger}_{-\mathbf{p}} b_{\mathbf{k}}e^{-i(E_{\mathbf{k}}-E_{\mathbf{p}})t}) \right]\nonumber \\
	 & + \int d \widetilde{p} \: d \widetilde{k} \: (2 \pi)^3 \delta(\mathbf{k} + \mathbf{p})\left[ 
	 (-E_{\mathbf{p}} E_{\mathbf{k}} -\mathbf{p} \mathbf{k} + m^2)
	  ( b^{\dagger}_{-\mathbf{p}} a^{\dagger}_{-\mathbf{k}} e^{i (E_{\mathbf{p}}+E_{\mathbf{k}})t} +a_{\mathbf{p}}b_{\mathbf{k}} e^{-i (E_{\mathbf{p}}+E_{\mathbf{k}})t})
	\right] \nonumber \\
	&= \int \: \frac{d^3 k }{(2 \pi)^3 \: 2 E_{\mathbf{k}}} \:\left[ 
	 (E_{\mathbf{k}} E_{\mathbf{k}} +\mathbf{k} \mathbf{k} + m^2)
	 (a_{-\mathbf{k}} a^{\dagger}_{-\mathbf{k}} + b^{\dagger}_{\mathbf{k}} b_{\mathbf{k}}) \right] \nonumber \\
	&=\int \: \frac{d^3 k }{(2 \pi)^3 \: 2 E_{\mathbf{k}}} \:\left[ 
	 (2E_{\mathbf{k}}^2)
	 (a_{-\mathbf{k}} a^{\dagger}_{-\mathbf{k}} + b^{\dagger}_{\mathbf{k}} b_{\mathbf{k}}) \right] \nonumber \\
	&=\int \: \frac{d^3 k }{(2 \pi)^3 } \:
	 E_{\mathbf{k}}
	 (a_{-\mathbf{k}} a^{\dagger}_{-\mathbf{k}} + b^{\dagger}_{\mathbf{k}} b_{\mathbf{k}})  \nonumber \\
	&=\int \: \frac{d^3 k }{(2 \pi)^3 } \:
	 E_{\mathbf{k}}
	 ( a^{\dagger}_{-\mathbf{k}} a_{-\mathbf{k}}+ b^{\dagger}_{\mathbf{k}} b_{\mathbf{k}} +1)  \nonumber
\end{align}
且易证明$\int \frac{d^3 k }{(2 \pi)^3 } E_{\mathbf{k}}\:a^{\dagger}_{-\mathbf{k}} a_{-\mathbf{k}} = \int \frac{d^3 k }{(2 \pi)^3 } E_{\mathbf{k}}\:a^{\dagger}_{\mathbf{k}} a_{\mathbf{k}}$ 且 将上式写成正规乘积得形式,将发散的常数项去掉,最终有
$$H = \int \: \frac{d^3 k }{(2 \pi)^3 } \:
	 E_{\mathbf{k}}
	 ( a^{\dagger}_{\mathbf{k}} a_{\mathbf{k}}+ b^{\dagger}_{\mathbf{k}} b_{\mathbf{k}}) $$

\bf{(c)}
$$
Q=\int d^3 x  \frac{i}{2} \left( \phi^{*} \pi^{*} - \pi \phi \right)
$$
带入上述表达式,有如下表示
\begin{align}
Q &= \frac{i}{2} \int d^3 x \: 
	d \widetilde{p} \: d \widetilde{k} \: e^{i (\mathbf{p}+\mathbf{k}) \mathbf{x}}
	\left(
	a^{\dagger}_{-\mathbf{p}} e^{i E_{\mathbf{p}}t} +  b_{\mathbf{p}} e^{-i E_{\mathbf{p}}t} 
	\right) 
	i E_{\mathbf{k}}
	\left(
	- a_{\mathbf{k}} e^{-i E_{\mathbf{k}}t} +  b^{\dagger}_{-\mathbf{k}} e^{i E_{\mathbf{k}}t} 
	\right) \nonumber \\
 & -\frac{i}{2} \int d^3 x \: 
	d \widetilde{p} \: d \widetilde{k} \: e^{i (\mathbf{p}+\mathbf{k}) \mathbf{x}}
	(i E_{\mathbf{p}}) 
	\left(
	a^{\dagger}_{-\mathbf{p}} e^{i E_{\mathbf{p}}t} -  b_{\mathbf{p}} e^{-i E_{\mathbf{p}}t} 
	\right) 
	\left(
	a_{\mathbf{k}} e^{-i E_{\mathbf{k}}t} +  b^{\dagger}_{-\mathbf{k}} e^{i E_{\mathbf{k}}t} 
	\right)   \nonumber \\
 &=-\frac{1}{2}  \: \int
	d \widetilde{p} \: d \widetilde{k} \: (2 \pi)^3 \delta(\mathbf{p}+\mathbf{k}) E_{\mathbf{k}}
	\left(
	a^{\dagger}_{-\mathbf{p}} e^{i E_{\mathbf{p}}t} +  b_{\mathbf{p}} e^{-i E_{\mathbf{p}}t} 
	\right) 
	\left(
	- a_{\mathbf{k}} e^{-i E_{\mathbf{k}}t} +  b^{\dagger}_{-\mathbf{k}} e^{i E_{\mathbf{k}}t} 
	\right) \nonumber \\
 & + \frac{1}{2} \:
 	d \widetilde{p} \: d \widetilde{k} \: (2 \pi)^3 \delta(\mathbf{p}+\mathbf{k}) E_{\mathbf{p}}
	\left(
	a^{\dagger}_{-\mathbf{p}} e^{i E_{\mathbf{p}}t} -  b_{\mathbf{p}} e^{-i E_{\mathbf{p}}t} 
	\right) 
	\left(
	a_{\mathbf{k}} e^{-i E_{\mathbf{k}}t} +  b^{\dagger}_{-\mathbf{k}} e^{i E_{\mathbf{k}}t} 
	\right)   \nonumber \\
 &= \frac{1}{2}  \: \int
	d \widetilde{p} \: d \widetilde{k} \: (2 \pi)^3 \delta(\mathbf{p}+\mathbf{k}) 
	\left(
	(E_{\mathbf{k}}+E_{\mathbf{p}})a^{\dagger}_{-\mathbf{p}} a_{\mathbf{k}} e^{i (E_{\mathbf{p}}-E_{\mathbf{k}})t} 
+	(E_{\mathbf{k}}-E_{\mathbf{p}})b_{\mathbf{p}} a_{\mathbf{k}} e^{-i(E_{\mathbf{p}}+E_{\mathbf{k}})t}
	\right) \nonumber \\
&+ \frac{1}{2}  \: \int
	d \widetilde{p} \: d \widetilde{k} \: (2 \pi)^3 \delta(\mathbf{p}+\mathbf{k}) 
	\left(
	(-E_{\mathbf{k}}+E_{\mathbf{p}})a^{\dagger}_{-\mathbf{p}}  b^{\dagger}_{-\mathbf{k}} e^{i (E_{\mathbf{p}}+E_{\mathbf{k}})t} 
-(E_{\mathbf{k}}+E_{\mathbf{p}}) b_{\mathbf{p}}  b^{\dagger}_{-\mathbf{k}}e^{-i (E_{\mathbf{p}}-E_{\mathbf{k}})t} 
	\right) \nonumber \\
&= \frac{1}{2} \int \: \frac{d^3 k }{(2 \pi)^3 } \:
	 (a^{\dagger}_{\mathbf{k}} a_{\mathbf{k}} - b_{\mathbf{-k}} b^{\dagger}_{\mathbf{-k}} )  \nonumber \\
&=\frac{1}{2} \int \: \frac{d^3 k }{(2 \pi)^3 } \:
	  \: N(a^{\dagger}_{\mathbf{k}} a_{\mathbf{k}} - b_{\mathbf{-k}} b^{\dagger}_{\mathbf{-k}})  \nonumber \\
&=\frac{1}{2} \int \: \frac{d^3 k }{(2 \pi)^3 } \:
	 (a^{\dagger}_{\mathbf{k}} a_{\mathbf{k}} - b^{\dagger}_{\mathbf{k}} b_{\mathbf{k}}  )  \nonumber 
\end{align}
且易得$[Q,H]=0$
\begin{align}
Q a^{\dagger}_{\mathbf{p}} \left | 0 \right \rangle &=\frac{1}{2} \int \: \frac{d^3 k }{(2 \pi)^3 } \:
	 (a^{\dagger}_{\mathbf{k}} a_{\mathbf{k}} - b^{\dagger}_{\mathbf{k}} b_{\mathbf{k}}  ) a^{\dagger}_{\mathbf{p}}
	 \left | 0 \right \rangle \nonumber \\
	 &=\frac{1}{2} \int \: \frac{d^3 k }{(2 \pi)^3 } \:
	 a^{\dagger}_{\mathbf{k}} a_{\mathbf{k}} a^{\dagger}_{\mathbf{p}}
	 \left | 0 \right \rangle \nonumber \\
	  &=\frac{1}{2} \int \: \frac{d^3 k }{(2 \pi)^3 } \:
	 a^{\dagger}_{\mathbf{k}}  (a^{\dagger}_{\mathbf{p}} a_{\mathbf{k}} +(2 \pi)^3 \delta(\mathbf{p}-\mathbf{k}))
	 \left | 0 \right \rangle \nonumber \\
 & =\frac{1}{2} a^{\dagger}_{\mathbf{p}} \left | 0 \right \rangle \nonumber 
\end{align}
同理可证明
$$Q b^{\dagger}_{\mathbf{p}} \left | 0 \right \rangle = -\frac{1}{2} b^{\dagger}_{\mathbf{p}} \left | 0 \right \rangle $$
由此,我们可以说明,$Q$为荷算符,生成一个正粒子,其带着有$\frac{1}{2}$的荷。生成一个反粒子,其带着有$-\frac{1}{2}$的荷。

\bf{(d)}
因为两粒子为全同粒子,则可以将产生湮灭算符的对易关系改写为
$[a_{\mathbf{p}\:a},a^{\dagger}_{\mathbf{k}\:b}] = (2 \pi)^3 \delta(\mathbf{p}-\mathbf{k}) \delta_{ab}$
首先考虑(c)情况,将两个粒子的情况带入得:
$$
Q=\frac{1}{2} \int \: \frac{d^3 k }{(2 \pi)^3 } \:
	 (a^{\dagger}_{\mathbf{k} \: a} a_{\mathbf{k} \: a} - b^{\dagger}_{\mathbf{k} \: a} b_{\mathbf{k} \: a}  )
$$
\begin{align}
Q^{i}&=\int d^3 x  \frac{i}{2} \left( 
		\phi^{*}_{a} (\sigma^{i})_{ab} \pi^{*}_{b} - \pi_{a} (\sigma^{i})_{ab} \phi_{b} 								\right) \nonumber \\
	&=\frac{i}{2} \int d^3 x \: 
	d \widetilde{p} \: d \widetilde{k} \: e^{i (\mathbf{p}+\mathbf{k}) \mathbf{x}}(i E_{\mathbf{k}})
	\left(
	a^{\dagger}_{-\mathbf{p}\:a} e^{i E_{\mathbf{p}}t} +  b_{\mathbf{p}\:a} e^{-i E_{\mathbf{p}}t} 
	\right) 
		(\sigma^{i})_{ab}
	\left(
	- a_{\mathbf{k} \:b} e^{-i E_{\mathbf{k}}t} +  b^{\dagger}_{-\mathbf{k} \:b} e^{i E_{\mathbf{k}}t} 
	\right) \nonumber \\
	 & -\frac{i}{2} \int d^3 x \: 
	d \widetilde{p} \: d \widetilde{k} \: e^{i (\mathbf{p}+\mathbf{k}) \mathbf{x}}(i E_{\mathbf{p}}) 
	\left(
	a^{\dagger}_{-\mathbf{p} \:a} e^{i E_{\mathbf{p}}t} -  b_{\mathbf{p} \:a} e^{-i E_{\mathbf{p}}t} 
	\right) 
	(\sigma^{i})_{ab}
	\left(
	a_{\mathbf{k} \:b} e^{-i E_{\mathbf{k}}t} +  b^{\dagger}_{-\mathbf{k} \:b} e^{i E_{\mathbf{k}}t} 
	\right)   \nonumber  \\
	&= \frac{1}{2} \int \: \frac{d^3 k }{(2 \pi)^3 } \:
	 (a^{\dagger}_{\mathbf{k} \: a} (\sigma^{i})_{ab} a_{\mathbf{k} \: b } - b_{\mathbf{k} \: a} (\sigma^{i})_{ab} b^{\dagger}_{\mathbf{k} \: b } )  \nonumber 
\end{align}
若要验证$Q^{i}$的性质与$SU(2)$相近,则,我们可以验证$[Q^{i},Q^{j}] = i \epsilon ^{ijk} Q^{k}$
%\begin{align}
%[Q^{i},Q^{j}] &= \frac{1}{4} \int \: \frac{d^3 k }{(2 \pi)^3 } \:\frac{d^3 p }{(2 \pi)^3 } \:
%	 [a^{\dagger}_{\mathbf{k} \: a} (\sigma^{i})_{ab} a_{\mathbf{k} \: b } - b_{\mathbf{k} \: a} (\sigma^{i})_{ab} b^%{\dagger}_{\mathbf{k} \: b },  
%	 a^{\dagger}_{\mathbf{p} \: c} (\sigma^{j})_{cd} a_{\mathbf{p} \: d } - b_{\mathbf{p} \: c} (\sigma^{j})_{cd} b^{%\dagger}_{\mathbf{p} \: d }] \nonumber \\
%%	 &=\frac{1}{4} \int \: \frac{d^3 k }{(2 \pi)^3 } \:\frac{d^3 p }{(2 \pi)^3 } \:
%%	 [a^{\dagger}_{\mathbf{k} \: a} (\sigma^{i})_{ab} a_{\mathbf{k} \: b } ,  
%%	 a^{\dagger}_{\mathbf{p} \: c} (\sigma^{j})_{cd} a_{\mathbf{p} \: d } ] \nonumber \\
%%	 &+\frac{1}{4} \int \: \frac{d^3 k }{(2 \pi)^3 } \:\frac{d^3 p }{(2 \pi)^3 } \:
%%	 [ b_{\mathbf{k} \: a} (\sigma^{i})_{ab} b^{\dagger}_{\mathbf{k} \: b },  
%%	  b_{\mathbf{p} \: c} (\sigma^{j})_{cd} b^{\dagger}_{\mathbf{p} \: d }] \nonumber \\
%%	 &=\frac{1}{4} \int \: \frac{d^3 k }{(2 \pi)^3 } \:d^3 p   \:
%%	 [a^{\dagger}_{\mathbf{k} \: a} (\sigma^{i})_{ab}\delta(\mathbf{k} -\mathbf{p}) \delta_{bc}(\sigma^{j})_{cd} a_{\mathbf{p} \: d }  
%%	 -a^{\dagger}_{\mathbf{p} \: c} (\sigma^{j})_{cd}\delta(\mathbf{k} -\mathbf{p}) \delta_{ad}(\sigma^{i})_{ab} a_{\mathbf{k} \: b }] \nonumber \\
%	 &-\frac{1}{4} \int \: \frac{d^3 k }{(2 \pi)^3 } \:d^3 p   \:
%	 [ b_{\mathbf{k} \: a} (\sigma^{i})_{ab} \delta(\mathbf{k} -\mathbf{p}) \delta_{bc} (\sigma^{j})_{cd} b^{\dagger}_{\mathbf{p} \: d }
%	 - b_{\mathbf{p} \: c} (\sigma^{j})_{cd}\delta(\mathbf{k} -\mathbf{p}) \delta_{ad}(\sigma^{i})_{ab} b^{\dagger}_{\mathbf{k} \: b }] \nonumber \\
%	 &= \frac{1}{4} \int \: \frac{d^3 k }{(2 \pi)^3 } \:d^3 p   \:
%	 [a^{\dagger}_{\mathbf{k} \: a} (\sigma^{i})_{ab} (\sigma^{j})_{bd} a_{\mathbf{k} \: d }  
%	 -a^{\dagger}_{\mathbf{k} \: c} (\sigma^{j})_{ca} (\sigma^{i})_{ab} a_{\mathbf{k} \: b }] \nonumber \\
%	 &-\frac{1}{4} \int \: \frac{d^3 k }{(2 \pi)^3 } \:d^3 p   \:
%	 [ b_{\mathbf{k} \: a} (\sigma^{i})_{ab} (\sigma^{j})_{bd} b^{\dagger}_{\mathbf{k} \: d }
%	 - b_{\mathbf{k} \: c} (\sigma^{j})_{ba} (\sigma^{i})_{ab} b^{\dagger}_{\mathbf{k} \: b }] \nonumber \\
%	 &=
%\end{align}
\begin{align}
[Q^{i},Q^{j}] &= -\frac{1}{4}\int \: d^3 x \: d^3 x^{\prime}[\phi^{*}_{a} (\sigma^{i})_{ab} \pi^{*}_{b} - \pi_{a} 
				(\sigma^{i})_{ab} \phi_{b}, \phi^{*}_{c} (\sigma^{j})_{cd} \pi^{*}_{d} - \pi_{c} (\sigma^{j})_{cd} \phi_{d} ] \nonumber \\
			  &=-\frac{1}{4}\int \: d^3 x \: d^3 x^{\prime} \:
			   (\sigma^{i})_{ab} (\sigma^{j})_{cd}
			   [\phi^{*}_{a}(x)\pi^{*}_{b}(x),\phi^{*}_{c}(x^{\prime})  \pi^{*}_{d}(x^{\prime})] \nonumber \\
			  &+\frac{1}{4}\int \: d^3 x \: d^3 x^{\prime} \:
			  (\sigma^{i})_{ab} (\sigma^{j})_{cd}
			  [\pi_{a}(x)  \phi_{b}(x),\phi^{*}_{c}(x^{\prime})  \pi^{*}_{d}(x^{\prime})] \nonumber \\
			  &+\frac{1}{4}\int \: d^3 x \: d^3 x^{\prime} \:
			  (\sigma^{i})_{ab} (\sigma^{j})_{cd} 
			  [\phi^{*}_{a}(x)  \pi^{*}_{b}(x), \pi_{c}(x^{\prime})  \phi_{d}(x^{\prime}) ] \nonumber \\
			  &-\frac{1}{4}\int \: d^3 x \: d^3 x^{\prime} \:
			  (\sigma^{i})_{ab} (\sigma^{j})_{cd}
			  [\pi_{a}(x)  \phi_{b}(x), \pi_{c}(x^{\prime})  \phi_{d}(x^{\prime})] \nonumber
\end{align}

拿其中一项为例:
\begin{align}
[\phi^{*}_{a}(x)\pi^{*}_{b}(x),\phi^{*}_{c}(x^{\prime})  \pi^{*}_{d}(x^{\prime})] 
	&= \phi^{*}_{c}(x^{\prime}) [\phi^{*}_{a}(x), \pi^{*}_{d}(x^{\prime})]\pi^{*}_{b}(x) \nonumber \\
	&+\phi^{*}_{a}(x)[\pi^{*}_{b}(x), \phi^{*}_{c}(x^{\prime})]\pi^{*}_{d}(x^{\prime}) \nonumber \\
	&= \phi^{*}_{c}(x^{\prime}) \delta(\mathbf{x} - \mathbf{x}^{\prime}) \delta_{ad} \pi^{*}_{b}(x) \nonumber \\
	&-\phi^{*}_{a}(x)\delta(\mathbf{x} - \mathbf{x}^{\prime}) \delta_{bc}\pi^{*}_{d}(x^{\prime}) \nonumber 
\end{align}

则原表达式且利用
$$\sigma^{i} \sigma^{j}=\frac{1}{2} ( \{ \sigma^{i},\sigma^{j} \} +[\sigma^{i},\sigma^{j}] ) = \delta^{ij} 
+ i \epsilon^{ij}_{\phantom{ij} k} \sigma^{k}$$
\begin{align}
[Q^{i},Q^{j}] 
&= \frac{1}{4} \int d^3 x \:
(\sigma^{j})_{ca} (\sigma^{i})_{ab} [\phi_{b} \pi_{c} - \pi^{*}_{b}\pi^{*}_{c}]
+  (\sigma^{i})_{ab} (\sigma^{j})_{bd}[ \phi^{*}_{a}\pi^{*}_{d}- \phi_{a} \pi_{d}]
\nonumber \\
&=\frac{1}{4} \int d^3 x \:
(\delta^{ij}-i \epsilon^{ij}_{\phantom{ij} k} \sigma^{k})_{cb} [\phi_{b} \pi_{c} - \pi^{*}_{b}\pi^{*}_{c}] \nonumber \\
&+\frac{1}{4} \int d^3 x \:
(\delta^{ij}+i \epsilon^{ij}_{\phantom{ij} k} \sigma^{k})_{ad} [\phi^{*}_{a}\pi^{*}_{d}- \phi_{a} \pi_{d}] \nonumber \\
&=i \epsilon^{ij}_{\phantom{ij} k} \int d^3 x  \frac{i}{2} \left( 
		\phi^{*}_{a} (\sigma^{k})_{ab} \pi^{*}_{b} - \pi_{a} (\sigma^{k})_{ab} \phi_{b} 								\right) \nonumber \\
&=i \epsilon^{ij}_{\phantom{ij} k} Q^{k} \nonumber \\
\end{align}
考虑两个复标量场
\begin{align}
\mathcal{L} &= \frac{1}{2} \partial_{\mu}(Re \phi_1) \partial^{\mu}(Re \phi_1) - \frac{1}{2} m^2 (Re \phi_1)^2 + (Re \rightarrow Im, 1 \rightarrow 2 ) \nonumber \\
&= \left(
	\begin{matrix}
		\phi_1^{*} & \phi_2^{*} \\
	\end{matrix} 
\right)
\left(
	\begin{matrix}
		-\partial^2 - m^2 & 0 \\
		0 & -\partial^2 - m^2 \\
	\end{matrix} 
\right)
\left(
	\begin{matrix}
		\phi_1 \\
		\phi_2\\
	\end{matrix} 
\right)
\end{align}
且在$SU(2)$的表示空间下不变 \\
经过推广,可以得到,在$U(n)$表示空间下,存在一个N维的场量
单个复标量场满足$U(1)$对称性,我们可以假设N个全同粒子满足$U(N)$对称性。N个复标量场形成一个对称不变空间
生成元有$N^2$个。带入二个粒子情况,在(d)中,我们可以知道有三个生成元存在,为SU(2)的生成元(共$N^2-1$),再加上(c)问中的单位矩阵
即构成了U(2)群中的四个生成元,并如下形式:
$$Q^{\mu} =\int d^3 x  \frac{i}{2} \left( 
		\phi^{*}_{a} (\sigma^{\mu})_{ab} \pi^{*}_{b} - \pi_{a} (\sigma^{\mu})_{ab} \phi_{b} 								\right)$$
我们可以根据其为$U(n)$,做出推广
设:
$$
\Phi = (\phi_1, \phi_2, \ldots ,\phi_n) \quad \Phi^{\prime} = e^{-i G r} \Phi
$$
\begin{align}
Q^{ij} &= \int d^3 x \: (\frac{\partial \mathcal{L}}{\partial(\partial_{\mu} \Phi)} \delta \Phi +
		\frac{\partial \mathcal{L}}{\partial(\partial_{\mu} \Phi^{\dagger})} \delta \Phi^{\dagger} )\nonumber \\
		&=\int d^3 x \: (\frac{\partial \mathcal{L}}{\partial(\partial_{\mu} \Phi)} (e^{-i G r} - 1 )\Phi +
		\frac{\partial \mathcal{L}}{\partial(\partial_{\mu} \Phi^{\dagger})} (e^{i G r} - 1 ) \Phi^{\dagger}) \nonumber \\
       & = A \: i \int d^3 x \left( 
		\phi^{*}_{a} (G^{ij})_{ab} \pi^{*}_{b} - \pi_{a} (G^{ij})_{ab} \phi_{b} \right) \nonumber 
\end{align}
A是某个与N有关的参数
\section{}
\bf{(a)}
由
$$
\left[J^{\mu \nu}, J^{\rho \sigma} \right] = i(g^{\nu \rho} J^{\mu \sigma} - g^{\mu \rho} J^{\nu \sigma}-g^{\nu \sigma} J^{\mu \rho}+g^{\mu \sigma} J^{\nu \rho}) $$
$$
L^{i} = \frac{1}{2} \epsilon^{ijk} J^{jk},\quad K^{i}=J^{0i} $$
$$
J^{\mu \nu} = - J^{\nu \mu}$$
得
$$
L^{1} = \frac{1}{2} (J^{23}-J^{32}) = J^{23}, \quad
L^{2} = J^{31}, \quad
L^{3} = J^{12} 
$$
\begin{align}
\left[L^1, L^2 \right] 	&=\left[J^{2 3}, J^{3 1} \right] \nonumber \\
						&=i(g^{3 3} J^{2 1} - g^{2 3} J^{3 1}-g^{3 1} J^{2 3}+g^{2 1} J^{3 3}) \nonumber \\
						&=-iJ^{2 1} \nonumber \\
						&= iJ^{12} = iL^3 \nonumber \\
\left[L^2, L^3 \right] 	&=\left[J^{3 1}, J^{1 2}  \right] \nonumber \\
						&= i(g^{1 1} J^{3 2} - g^{3 1} J^{1 2}-g^{1 2} J^{3 1}+g^{3 2} J^{1 1}) \nonumber \\
						&=-iJ^{32} \nonumber \\
						&= iJ^{23} = iL^1 \nonumber \\
\left[L^3, L^1 \right] 	&=\left[J^{1 2}, J^{2 3} \right] \nonumber \\
						&= i(g^{2 2} J^{1 3} - g^{1 2} J^{2 3}-g^{2 3} J^{1 2}+g^{1 3} J^{2 2}) \nonumber \\
						&=-iJ^{13} \nonumber \\
						&= iJ^{31} = iL^2 \nonumber 
\end{align}
\begin{align}
\left[K^1, K^2 \right] 	&=\left[J^{0 1}, J^{0 2} \right] \nonumber \\
						&= i(g^{1 0} J^{0 2} - g^{0 0} J^{1 2}-g^{1 2} J^{0 0}+g^{0 2} J^{1 0}) \nonumber \\
						&=-iJ^{12} = -L^{3} \nonumber \\
\left[K^2, K^3 \right] 	&=\left[J^{0 2}, J^{0 3} \right] \nonumber \\
						&= i(g^{2 0} J^{0 3} - g^{0 0} J^{2 3}-g^{2 3} J^{0 0}+g^{0 3} J^{2 0})\nonumber \\
						&=-iJ^{23} = -L^{1} \nonumber \\
\left[K^3, K^1 \right] 	&=\left[J^{0 3}, J^{0 1} \right] \nonumber \\
						&= i(g^{3 0} J^{0 1} - g^{0 0} J^{3 1}-g^{3 1} J^{0 0}+g^{0 1} J^{3 0}) \nonumber \\
						&=-iJ^{31} = -L^{2} \nonumber 
\end{align}
\begin{align}
\left[L^1, K^2 \right]	&=\left[J^{2 3}, J^{0 2} \right] \nonumber \\
						&= i(g^{3 0} J^{2 2} - g^{2 0} J^{3 2}-g^{3 2} J^{2 0}+g^{2 2} J^{3 0})	\nonumber \\
						&= - iJ^{30} = iK^{3} = \left[K^2, L^1 \right] \nonumber \\	
\left[L^2, K^3 \right]	&=\left[J^{3 1}, J^{0 3} \right] \nonumber \\	
						&= i(g^{1 0} J^{3 3} - g^{3 0} J^{1 3}-g^{1 3} J^{3 0}+g^{3 3} J^{1 0}) \nonumber \\
						&= - iJ^{10} = iK^{1} = \left[K^3, L^2 \right] \nonumber \\		
\left[L^3, K^1 \right]	&=\left[J^{1 2}, J^{0 1} \right] \nonumber \\	
						&= i(g^{2 0} J^{1 1} - g^{1 0} J^{2 1}-g^{2 1} J^{1 0}+g^{1 1} J^{2 0})  \nonumber \\
						&= - iJ^{20} = iK^{2} = \left[K^1, L^3 \right] \nonumber 	
\end{align}
由上式可以得出
$$
\left[ L^{i}, L^{j} \right] = i \epsilon^{ijk} L^{k} \quad \left[ K^{i}, K^{j} \right] = i \epsilon^{ijk} L^{k}
\quad \left[ L^{i}, K^{j} \right]=\left[ K^{i},L^{j}  \right] = -i \epsilon^{ijk} K^{k}
$$
考虑构造$$
\mathbf{J}_{+} = \frac{1}{2}(\mathbf{L} + i \mathbf{K}) \quad \mathbf{J}_{-} = \frac{1}{2}(\mathbf{L} - i \mathbf{K}) 
$$
\begin{align}
\left[J_{+}^{i} ,J_{+}^{j} \right] 
			&= \left[\frac{1}{2}(L^{i} + i K^{i}) ,\frac{1}{2}(L^{j} + i K^{j}) \right] \nonumber \\
			&= \frac{1}{4} (\left[ L^{i},L^{j} \right] + i\left[ K^{i},L^{j} \right] 
				+i\left[ L^{i},K^{j} \right] -\left[ K^{i},K^{j} \right]) \nonumber \\
			&=i \epsilon^{ijk} \frac{1}{4}(L^{k} + 2i K^{k} - (- L^{k})) \nonumber \\
			&=i  \epsilon^{ijk} \frac{1}{4}(2L^{k}+2iK^{k}) \nonumber \\
			&=i \epsilon^{ijk} J_{+}^{k}  \\ \nonumber
\left[J_{-}^{i} ,J_{-}^{j} \right] 
			&= \left[\frac{1}{2}(L^{i} - i K^{i}) ,\frac{1}{2}(L^{j} - i K^{j}) \right] \nonumber \\
			&= \frac{1}{4} (\left[ L^{i},L^{j} \right] - i\left[ K^{i},L^{j} \right] 
				-i\left[ L^{i},K^{j} \right] -\left[ K^{i},K^{j} \right]) \nonumber \\
			&=i \epsilon^{ijk} \frac{1}{4}(L^{k} - 2i K^{k}- (-L^{k})) \nonumber \\
			&=i  \epsilon^{ijk} \frac{1}{4}(2L^{k}-2iK^{k}) \nonumber \\
			&=i \epsilon^{ijk} J_{-}^{k} \\ \nonumber
\left[J_{+}^{i} ,J_{-}^{j} \right] 
			&= \left[\frac{1}{2}(L^{i} + i K^{i}) ,\frac{1}{2}(L^{j} - i K^{j}) \right] \nonumber \\
			&= \frac{1}{4} (\left[ L^{i},L^{j} \right] + i\left[ K^{i},L^{j} \right] 
				-i\left[ L^{i},K^{j} \right] +\left[ K^{i},K^{j} \right]) \nonumber \\
			&=i \epsilon^{ijk} \frac{1}{4}(L^{k} + 0 + (-L^{k})) \nonumber \\
			&=0\nonumber
\end{align}

\bf{(b)}

1.在$(\frac{1}{2},0)$表示里面。$$J_{+}^{i}=\frac{\sigma^{i}}{2}, \quad J_{-}^{i} = 0 $$
于是有
\begin{align}
 L^{i} &= J_{+}^{i} + J_{-}^{i} = \frac{\sigma^{i}}{2} \nonumber \\
 K^{i} &= -i(J_{+}^{i} - J_{-}^{i}) = -i \frac{\sigma^{i}}{2} \nonumber 
\end{align}
由(a)中无穷小Lorentz变换可写为
$$U = 1- i \boldsymbol{\theta} \cdot \mathbf{L} -i\boldsymbol{\beta} \cdot \mathbf{K}
=1- i \boldsymbol{\theta} \cdot \frac{\boldsymbol{\sigma}}{2} -\boldsymbol{\beta} \cdot \frac{\boldsymbol{\sigma}}{2}$$
则$\psi_{L}$满足$\psi_{L} \rightarrow (1- i \boldsymbol{\theta} \cdot \frac{\boldsymbol{\sigma}}{2} -\boldsymbol{\beta} \cdot \frac{\boldsymbol{\sigma}}{2})\psi_{L}$
\\同样地在右手旋量空间有:\\
2.在$(0,\frac{1}{2})$表示里面。$$J_{-}^{i}=\frac{\sigma^{i}}{2}, \quad J_{+}^{i} = 0 $$
于是有
\begin{align}
 L^{i} &= J_{+}^{i} + J_{-}^{i} = \frac{\sigma^{i}}{2} \nonumber \\
 K^{i} &= -i(J_{+}^{i} - J_{-}^{i}) = i \frac{\sigma^{i}}{2} \nonumber 
\end{align}
由(a)中无穷小Lorentz变换可写为
$$U = 1- i \boldsymbol{\theta} \cdot \mathbf{L} - i\boldsymbol{\beta}  \cdot \mathbf{K}
=1- i \boldsymbol{\theta} \cdot \frac{\boldsymbol{\sigma}}{2} +\boldsymbol{\beta} \cdot \frac{\boldsymbol{\sigma}}{2}$$
则$\psi_{R}$满足$\psi_{R} \rightarrow (1- i \boldsymbol{\theta} \cdot \frac{\boldsymbol{\sigma}}{2} +\boldsymbol{\beta} \cdot \frac{\boldsymbol{\sigma}}{2})\psi_{R}$

\bf(c)
该题还可以通过 $SL(2,C)$群的一些性质很容易得到,如下的表达式\\
在$(\frac{1}{2},\frac{1}{2})$表示之中,我们有二维的Lorentz变换,进行左右手旋量空间的转换。
$$
V = 
\left(
	\begin{matrix}
		V^0 + V^3 & V^1 - iV^2 \\
		V^1 + iV^2 & V^0 - V^3 \\
	\end{matrix} 
\right) = V^{0} + \mathbf{V} \cdot \boldsymbol{\sigma} = V^{\mu} \overline{\sigma}_{\mu}
$$
其中$V^{\mu} = (V^{0}, \mathbf{V}) \quad \overline{\sigma}_{\mu}=(1,\boldsymbol{\sigma})  $ \\
并且运用在量子场论课堂上在证明Lorentz 矢量时,推导的公式:
$$
-\frac{1}{2}[\boldsymbol{\sigma},\overline{\sigma}_{\mu}] = (\mathbf{L})_{\mu}^{\phantom{\mu} \nu} \overline{\sigma}_{\nu}
$$
$$
\frac{1}{2}\{\boldsymbol{\sigma},\overline{\sigma}_{\mu}\} = (i\mathbf{K})_{\mu}^{\phantom{\mu} \nu} \overline{\sigma}_{\nu}
$$
\begin{align}
V^{\mu} \overline{\sigma}_{\mu}  \rightarrow &V^{\prime \mu} \overline{\sigma}_{\mu}  \nonumber \\
  &=(1- i \boldsymbol{\theta} \cdot \frac{\boldsymbol{\sigma}}{2} +\boldsymbol{\beta} \cdot \frac{\boldsymbol{\sigma}}{2}) V (1+ i \boldsymbol{\theta} \cdot \frac{\boldsymbol{\sigma}}{2} +\boldsymbol{\beta} \cdot \frac{\boldsymbol{\sigma}}{2}) \nonumber \\
  &= (1- i \boldsymbol{\theta} \cdot \frac{\boldsymbol{\sigma}}{2} +\boldsymbol{\beta} \cdot \frac{\boldsymbol{\sigma}}{2}) V^{\mu} \overline{\sigma}_{\mu}(1+ i \boldsymbol{\theta} \cdot \frac{\boldsymbol{\sigma}}{2} +\boldsymbol{\beta} \cdot \frac{\boldsymbol{\sigma}}{2}) \nonumber \\
  &= V^{\mu} \overline{\sigma}_{\mu} - i \frac{\theta}{2} V^{\mu} [\boldsymbol{\sigma},\overline{\sigma}_{\mu}] + \frac{\beta}{2} V^{\mu} \{\boldsymbol{\sigma},\overline{\sigma}_{\mu}\} + \mathcal{O}(2) \nonumber \\
   &=V^{\mu} \overline{\sigma}_{\nu}[\delta_{\mu}^{\phantom{\mu} \nu}  + (i \boldsymbol{\theta} \cdot \mathbf{L}+i\boldsymbol{\beta} \cdot \mathbf{K})_{\mu}^{\phantom{\mu} \nu} ] \nonumber \\
   &=(e^{i \boldsymbol{\theta} \cdot \mathbf{L}+i\boldsymbol{\beta} \cdot \mathbf{K}})_{\mu}^{\phantom{\mu} \nu}  V^{\mu} \overline{\sigma}_{\nu} \nonumber \\
   &=\Lambda_{\mu}^{\phantom{\mu} \nu} V^{\mu} \overline{\sigma}_{\nu} \nonumber \\
   &=\Lambda_{\nu}^{\phantom{\nu} \mu} V^{\nu} \overline{\sigma}_{\mu} \nonumber 
\end{align}
$$
V^{\mu} \rightarrow V^{\prime \mu} =\Lambda_{\nu}^{\phantom{\nu} \mu} V^{\nu} = (\Lambda^{-1})^{\mu}_{\phantom{\mu} \nu} V^{\nu}
$$
与lorentz矢量所满足的变换吻合
$$T^{\mu} \rightarrow T^{\prime \mu} = \Lambda^{\mu}_{\phantom{\mu}\nu} T^{\nu}$$

\end{document}
